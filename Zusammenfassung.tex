%%%%%%%%%%%%%%%%%%%%%%%%%%%%%%%%%%%%%%%%%%%%%%%%%%%%%%%%%%%%%%%%%%%%%%%%%
%%%%%%%%%%%%%%%%%%%%%%%%%% Zusammenfassung  %%%%%%%%%%%%%%%%%%%%%%%%%%%%%
%%%%%%%%%%%%%%%%%%%%%%%%%%%%%%%%%%%%%%%%%%%%%%%%%%%%%%%%%%%%%%%%%%%%%%%%%
\chapter{Zusammenfassung} 

Das wesentliche Ziel dieser Arbeit war die Konstruktion einer beheizbaren Clusterquelle für Flüssigkeiten. Die Quelle wird im Rahmen von Experimenten mit Wasserdimeren an FLASH in Hamburg benötigt. \\

Die vorliegende Arbeit ist in drei Teile gegliedert.

In einem einführendem Kapitel sind die Grundlagen der Entstehung von Clustern in einer Überschallexpansion dargestellt. Darüberhinaus wurde in diesem Kapitel das Reaktionsmikroskop vorgestellt, das zur Analyse der Experimente mit Wasserdimeren verwendet wird. Dabei wurden Aufbau und Funktionsweise der wichtigsten Komponenten des Reaktionsmikroskopes, sowie die physikalischen Zusammenhänge der Clusterbildung ausführlich behandelt.

\textbf{Das zweite Kapitel begleitet den Leser von der Expansion von Gas ins Vakuum bis zur Entstehung eines experimentell nutzbaren kalten Teilchenstroms (?)}
Das zweite Kapitel erläutert die technische Umsetzung der theoretischen Grundlagen in ein tatsächliches Experiment. In diesem Zusammenhang wird die neu \textbf{entwickelte} Cluster-Quelle in Hinsicht der an sie gestellten Anforderungen und Möglichkeiten \textbf{präsentiert/eingehend besprochen}. Weiterhin werden die technischen Komponenten des experimentellen Ausbaus erläutert, die notwendig sind um einen kalten Targetjet (zu erzeugen/ in der Reaktionskammer zu erhalten/ zu leiten).

Das letzte Kapitel befasst sich mit Ergebnissen aus früheren Experimenten mit Wasserclustern an FLASH in Hamburg und der Inbetriebnahme der neu entwickelten Cluster-Quelle.
Im ersten Teil des Kapitels wurden die experimentellen Daten in Form von Flugzeitspektren und Ortbildern dargestellt. Die Analyse der Flugzeitspektren ergab, wie zu erwarten, eine deutliche Dominanz von Wasserionen. Relativ zur Flugzeit der Wasserionen, konnten alle relevanten Peaks aus dem Spektrum Ionen zugeordnet werden. Der Anteil der erzeugten Wasserdimere beschränkte sich hierbei auf 0,21 \% der aus dem Jet entstandenen Ionen.
Der zweite Teil des Kapitels veranschaulicht die Vorbereitungen auf den experimentellen Einsatz der Cluster-Quelle. \textbf{Auswirkungen der (Positionierung der Düse zum Skimmer/ der Lage von Düse zu Skimmer) auf die Strahlqualität werden quantitativ untersucht.} Zusätzlich wurde eine Charakterisierung der Quelle in Hinsicht auf ihr Heizverhalten durchgeführt. Dabei stellte sich heraus, dass bei konstanter Heizleistung nach drei Stunden 90 \% der Betriebstemperatur erreicht werden. Beginnt man jedoch mit hoher Heizleistung und reguliert diese nach erreichen einer Schwelltemperatur nach unten, kann nach weniger als 60 Minuten 98 \% der Zieltemperatur erreicht werden. Das Abkühlen der Quelle von Betriebstemperatur kann nicht beschleunigt werden. Diese verhält sich gleich wie das Heizen mit konstanter Leistung, sodass nach drei Stunden 90 \% der Temperaturdifferenz zur Raumtemperatur überwunden wurden. 

Für den Einsatz der neuen Clusterquelle an FLASH gilt es den Heizvorgang weiter zu optimieren und das Verhalten der Quelle im Einsatz mit Wasser zu untersuchen.

