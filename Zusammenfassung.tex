%%%%%%%%%%%%%%%%%%%%%%%%%%%%%%%%%%%%%%%%%%%%%%%%%%%%%%%%%%%%%%%%%%%%%%%%%
%%%%%%%%%%%%%%%%%%%%%%%%%% Zusammenfassung  %%%%%%%%%%%%%%%%%%%%%%%%%%%%%
%%%%%%%%%%%%%%%%%%%%%%%%%%%%%%%%%%%%%%%%%%%%%%%%%%%%%%%%%%%%%%%%%%%%%%%%%
\chapter{Zusammenfassung} 

Das Ziel dieser Arbeit war die Konstruktion einer beheizbaren Clusterquelle für Flüssigkeiten. Die Quelle wird im Rahmen von Experimenten mit Wasserdimeren für das Reaktionsmikroskop am Freie-Elektronen Laser in Hamburg (FLASH) benötigt. \\

In einem einführendem Kapitel wurden die Grundlagen der Entstehung von Clustern durch Überschallexpansion dargestellt. \\Darüberhinaus wurde in diesem Kapitel das Reaktionsmikroskop vorgestellt. Dabei wurden Aufbau und Funktionsweise der wichtigsten Komponenten des Reaktionsmikroskopes, sowie die physikalischen Zusammenhänge der Clusterbildung ausführlich behandelt.

Das zweite Kapitel erläuterte die technische Umsetzung der theoretischen Grundlagen. In diesem Zusammenhang wurde die neu entwickelte Cluster-Quelle in Hinsicht der an sie gestellten Anforderungen und Möglichkeiten eingehend besprochen. Desweiteren wurden die technischen Komponenten des experimentellen Ausbaus erläutert, die notwendig sind, um einen kalten Targetjet zu erzeugen.

Das letzte Kapitel befasste sich mit Ergebnissen aus früheren Experimenten mit Wasserclustern an FLASH und der Inbetriebnahme der neu entwickelten Cluster-Quelle.
Im ersten Teil des Kapitels wurden die experimentellen Daten in Form von Flugzeitspektren und Ortbildern untersucht. Die Analyse der Flugzeitspektren ergab eine deutliche Dominanz von Wasserionen. Relativ zur Flugzeit der Wasserionen konnten relevante Peaks aus dem Spektrum Ionenarten zugeordnet werden. Der Anteil der erzeugten Wasserdimere beschränkte sich hierbei auf 0,21 \%.
Der zweite Teil des Kapitels behandelt die Vorbereitungen auf den Einsatz der neu gebauten Cluster-Quelle. Auswirkungen der (Positionierung der Düse zum Skimmer auf die Strahlqualität wurden quantitativ untersucht. Zusätzlich wurde eine Charakterisierung der Quelle in Hinsicht auf ihr Heizverhalten durchgeführt. Dabei stellte sich heraus, dass bei konstanter Heizleistung nach drei Stunden 90 \% der Betriebstemperatur erreicht werden. Beginnt man jedoch mit hoher Heizleistung und reguliert diese nach erreichen einer Schwelltemperatur nach unten, kann nach weniger als 60 Minuten 98 \% der Zieltemperatur erreicht werden. Das Abkühlen der Quelle von Betriebstemperatur kann nicht beschleunigt werden. Diese verhält sich gleich wie das Heizen mit konstanter Leistung, sodass nach drei Stunden 90 \% der Temperaturdifferenz zur Raumtemperatur überwunden wurden. 

Für den Einsatz der neuen Clusterquelle an FLASH gilt es den Heizvorgang weiter zu optimieren und das Verhalten der Quelle im Einsatz mit Wasser zu untersuchen.

