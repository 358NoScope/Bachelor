
%%%%%%%%%%%%%%%%%%%%%%%%%%%%%%%%%%%%%%%%%%%%%%%%%%%%%%%%%%%%%%%%%%%%%%%%%
%%%%%%%%%%%%%%%%%%%%%%%%%%% M O T I V A T I O N %%%%%%%%%%%%%%%%%%%%%%%%%
%%%%%%%%%%%%%%%%%%%%%%%%%%%%%%%%%%%%%%%%%%%%%%%%%%%%%%%%%%%%%%%%%%%%%%%%%
  \chapter{Motivation} 
  
  Wasser ist auf der Erde ein allgegenwärtiges Molekül, das auch Grundbestandteil allen Lebens darstellt. Deswegen ist es sehr wichtig es so gut wie möglich zu untersuchen und zu verstehen. Wir haben mit dem Reaktionsmikroskop und dem FLASH in Hamburg gute Vorraussetzungen es genauer zu untersuchen.
  
  Das Ziel dieser Bachelorarbeit ist es einen Wasserjet zu erzeugen mit dem kleine Wassercluster untersucht werden können. Wir untersuchen hierbei kleine Cluster, weil diese leichter theoretisch zu berechnen sind und man somit erhaltene Ergebnisse gut mit der Theorie vergleichen kann. Am einfachsten zu berechnen sind natürlich Wasserdimere, weshalb wir auch gezielt diese Moleküle zu erzeugen versuchen.
  
  Interessante Anwendungsgebiete sind zum Beispiel die von [referenzGökhbergoderso] theoretisch behandelte steuerbare ICD. (Genauer beschreiben was das für ein Versuch ist) Mit einer Apparatur die Wasserdimere erzeugt(?), die man mit höhermassigen Atomen, z.B. Edelgasen, \enquote{dopt} und einem starken Laser, wie er uns am FLASH zur Verfügung steht, können solche Experimente durchgeführt werden. Mögliche Anwendungen in der Realität sind die Untersuchung von Auswirkungen der spontanen ICD auf Biomoleküle [Referenz von Kirsten nehmen]\\
  Eine weitere interessante Anwendung ist die Untersuchung von der proton transfer dynamic bei der Reaktion von\  $\mathrm{H}_2\mathrm{O}$ +  $\mathrm{H}_2\mathrm{O}^{+}$ \textrightarrow\ $\mathrm{H}_3^{+}\mathrm{O}$ + $\mathrm{H}\mathrm{O}$. Diese $\mathrm{H}_3^{+}\mathrm{O}$ - Radikale sind dafür bekannt Biomoleküle anzugreifen und in ihrer Struktur zu verändern. Auch dies kann man mithilfe der Wasserdimerquelle und einem Ionisierenden Laser mittels des Pump-Probe-Verfahrens zeitlich auflösen.\\
  \\
  Und dafür ist nun die Idee eben eine Jetdüse zu bauen, die ein internes Wasserreservoir besitzt und während des Betriebs wiederbefüllt werden kann. Die Düse soll 
  
  
  Am Ende der Motivation noch kurz die Inhaltsangabe machen
  
  
  %The radicals are known to react eciently with biomolecules, thereby modifying their structure (“damage”). A particular important example is the indirect DNA damage through hydroxyl radicals, which may originate from any free radical source in the close vicinity of DNA [13].
%\section{Versuche auf das Thema einzuführen, aber weis noch nicht genau wie ich das machen soll, weil meine Arbeit ist eher technisch und dann kann ich ja nicht so mit den coolen Sachen die du mir geschickt hast prahlen...}  
  

%Wasserdüse ist toll weil man noch nicht genug über Wasser weiß, und wir das mit Remi und FLASH genauer untersuchen können. Was auch ganz cool sein wird ist, dass wir in Zukunft gezielte ICD und Proton transfer dynamics untersuchen wollen.
%Dazu habe ich hier Wasserjet designt der wichtige Vorraussetzungen erfüllt.\\


%Ich habe vor eine Wasserdüse zu konstruieren und in Betrieb zu nehmen, die Wasserdimere produziert. Diese sollen in ein Remi als Target eingespeist werden und wir wollen dann wie erwähnt Icd triggern oder Proton transfer dynamics.

%Einleitung die den Leser abholt und langsam motiviert die Arbeit zu lesen. Interessante Punkte und Ziele vorstellen!

%Hier nochmal dasselbe als Kommentar, zum Vergleich vielleicht?
 
%Understanding and identifying the key reactions induced by the ionization of water on a molecular level is of great relevance for a variety of applications ranging from medical radiation therapy to nuclear reactor technologies [12]. Water is ubiquitous on earth and especially in the human body, which calls for a detailed analysis of its ionization dynamics. A particularly important reaction following the ionization of a water molecule is proton transfer from the initially produced cation to a neighboring molecule. As a result hydronium ions (H3O+) and hydroxyl radicals (HO) are formed: H2O+ + H2O ! H3O+ + HO: (1) The radicals are known to react efficiently with biomolecules, thereby modifying their structure (“damage”). A particular important example is the indirect DNA damage through hydroxyl radicals, which may originate from any free radical source in the close vicinity of DNA [13]. A hydronium ion, created within a water environment, may stabilize by forming a hydrogen bond to another water molecule: In this so-called Zundel ion (H5O+ 2 ) a proton is shared and oscillates between the neighbouring molecules [14,15]. Alternatively an Eigen ion (H3O+(H2O)3), the most common protonated water structure in bulk liquid, may be formed [16, 17]. Experimental studies on the photoionization of water clusters were so far mostly investigating the yields of dierent fragmentation channels, performed by time-of-flight measurements [18–20]. All of them observed protonated water clusters, which points towards a very fast rearrangement dynamics, initially induced by proton transfer.