
%%%%%%%%%%%%%%%%%%%%%%%%%%%%%%%%%%%%%%%%%%%%%%%%%%%%%%%%%%%%%%%%%%%%%%%%%
%%%%%%%%%%%%%%%%%%%%%%%%%%% M O T I V A T I O N %%%%%%%%%%%%%%%%%%%%%%%%%
%%%%%%%%%%%%%%%%%%%%%%%%%%%%%%%%%%%%%%%%%%%%%%%%%%%%%%%%%%%%%%%%%%%%%%%%%
  \chapter{Motivation} \label{sec:Motivation}
  
  \textbf{Siehe S 105 TuBerlinICDmit wasser2011design}
  
  Wasser ist auf der Erde ein allgegenwärtiges Molekül, das den Grundstein allen Lebens darstellt. Deswegen ist es eine Aufgabe der Wissenschaft, möglichst alle Vorgänge, die in und wegen Wasser vorkommen, bis ins Detail zu verstehen. Bereits gewonnenes Wissen darüber findet Anwendung in Bereichen, die von medizinischer Strahlentherapie, bis hin zu Kernkraftwerktechnologien reichen. 
  Auch diese Arbeit steht gewissermaßen unter einem medinischen Zeichen, denn das Ziel ist grundlegende Relaxationsvorgänge und Gleichgewichtsreaktionen in Wasser auf Grundlagenbasis untersuchen. Und da der menschliche Körper zu ca. 70\% aus Wasser besteht und grade die DNA sich in wässriger Umgebung befindet, ist es wichtig die genannten Vorgänge auch in Bezug auf die Wechselwirkung mit ihr zu untersuchen und mögliche Schlussfolgerungen für die Medizin zu ziehen.\\
  Bei Experimenten über die Fragmentation von Wasserclustern wurde beobachtet, dass sich erstaunlich oft protonisierte Wassercluster bilden, wie das Zundel-Ion ($\mathrm{H_5O_2^+}$), das Eigen-Ion ($\mathrm{H_3O^+(H_2O)_3}$) und den kleinsten Repräsentant davon, das Oxonium-Ion ($\mathrm{H_3O^+}$). Grundlegend für deren Entstehung ist der sogenannte Protonentransfer, über dessen Dynamik noch nicht viel bekannt ist. Es existieren allerdings eine Vielzahl von theoretischen Studien und Vorhersagen über die Protonentransferdynamik für verschiedene Größen von Wasserclustern. Natürlich steigt die Genauigkeit der Berechnungen mit sinkender Komplexität der Cluster, weswegen es von Vorteil ist, zuerst Wasserdimere in dieser Hinsicht zu untersuchen. Diese sind der beste Ansatzpunkt um Protonentransferdynamik zu untersuchen(\textbf{Wiederholung}), da für die Bildung des einfachsten protonierten Clusters, dem Oxonium-Ion, nur zwei Wassermoleküle erforderlich sind:
  \begin{equation}
  \textmd{{\footnotesize(Gleichgewichtsreaktion von Wasser) }}\qquad \textmd{H}_2\textmd{O} + \textmd{H}_2\textmd{O} \rightleftarrows \textmd{H}_3\textmd{O}^+ + \textmd{O}\textmd{H}^- 
  \end{equation} 
  
  Schafft man es die \textbf{Zeitskala?} des Protonentransfers an diesem Beispiel zu messen(\textbf{aufzulösen}), setzt man damit einen wichtigen Vergleichswert für die Theorie, da alle vorkommenden Protonentransfers darauf aufbauen. Schnorr $et\ al.$ von der Gruppe Pfeifer am MPIK Heidelberg haben vor, genau diese Messung am DESY in Hamburg durchzuführen \cite{SchPTD15}. Das DESY stellt den Freie-Elektronen-Laser in Hamburg (FLASH) zur Verfügung, mit dem XUV-Pulse einer Pulsdauer von wenigen 10 fs erzeugt werden können. Die \textbf{Dauer?} des Protonentransfers wird auf die selbe Größenordnung geschätzt und kann damit mit diesem Laser aufgelöst werden.
  %\textit{Noch mehr drauf eingehen? Mit Pump-Probe das ganze triggern und Flash liefert geeingneten Laser. Was dann noch fehlt ist geeignete Dimerquelle}*.
  Das Ziel dieser Bachelorarbeit ist es die noch fehlende Wasser-Jet-Quelle zu konstruieren, welche die für das Experiment benötigten Wasserdimere in angemessenem Anteil produziert. 
  \textit{Hier erwähnen dass schon versucht wurde, aber nicht praktikabel war oder gut funktioniert hat; Warum neue Düse?}
  Doch die Anwendungsmöglichkeiten der Wasserdüse sind nicht nur auf dieses eine Experiment beschränkt. Ein weiteres Beispiel findet man in dem Proposal von Schnorr $et\ al.$ \cite{SchICD15}. Im diesem wird die von Gokhberg et al \cite{gokhberg2014} veröffentlichte Idee, der gezielten Auslösung eines intermolekularen Coulomb-Zerfalls (englisch: intermolecular atomic decay, kurz: ICD) aufgegriffen. Unser Ziel wird es sein, in der entworfenen Wasserdüse, Wasser mit einem massereichen Edelgas (Xe) koexpandieren zu lassen um Xe-H$_2$O Moleküle zu formen. Diese sollen dann im Flash mit einem Laser zu ICD angeregt werden. Die Wellenlänge des Lasers muss dazu resonant zu einem bestimmten Übergang in Xenon sein. Dadurch, dass das Xenon einen viel größeren Wirkungsquerschnitt, als das Wasser hat (bei 100eV: 20Mbarn zu 1.8Mbarn) scheint es realistisch, das Xenon gezielt durch den Laser anzuregen, und den ICD-Prozess damit in Gang zu setzen. Und wenn die Wellenlänge nicht resonant verstimmt wird, soll kein ICD mehr stattfinden. Diese Technik wurde, u.a. auch von Gokhberg $et\ al.$ \cite{gokhberg2014}, vorgeschlagen zur Tumorbehandlung eingesetzt zu werden, denn die zweiten ICD-Elektronen besitzen meist eine Energie von unter 15 eV und sind dafür bekannt Molekülbindungen effizient zu zerstören.  Dazu müssten in der Praxis Zielmoleküle, mit einem den Querschnitt dominierenden Bestandteil, in die betroffene Region injiziert werden, um dort gezielt ICD anregen zu können. 
  Diese Technik würde der Medizin bei der Tumorbehandlung weiterhelfen, doch leider ist die experimentelle Erfahrung im ICD-Feld zurzeit noch weit von der praktischen Anwendung entfernt. Wenn das Experiment glückt, wäre es das erste Mal, dass ICD gezielt auf molekularer Ebene ausgelöst und gestoppt werden kann.
  

% Ab hier nur grobe Vorlage
%  
%  Interessante Anwendungsgebiete sind zum Beispiel die von [referenzGökhbergoderso] theoretisch behandelte steuerbare ICD. (Genauer beschreiben was das für ein Experiment ist) Mit einer Apparatur die Wasserdimere erzeugt(?), die man mit höhermassigen Atomen, z.B. Edelgasen, \enquote{dopt} und einem starken Laser, wie er am FLASH zur Verfügung steht, können solche Experimente durchgeführt werden. Mögliche Anwendungen in der Realität sind die Untersuchung von Auswirkungen der spontanen ICD auf Biomoleküle [Referenz von Kirsten nehmen]\\
%  Eine weitere interessante Anwendung ist die Untersuchung von der proton transfer dynamic bei der Reaktion von\  $\mathrm{H}_2\mathrm{O} +  \mathrm{H}_2\mathrm{O}^{+} \rightarrow  \mathrm{H}_3^{+}\mathrm{O} + \mathrm{H}\mathrm{O}$. Diese $\mathrm{H}_3^{+}\mathrm{O}$ - Radikale sind dafür bekannt Biomoleküle anzugreifen und in ihrer Struktur zu verändern. Auch dies kann man mithilfe der Wasserdimerquelle und einem Ionisierenden Laser mittels des Pump-Probe-Verfahrens zeitlich auflösen.\\
%  \\
% Und dafür ist nun die Idee eben eine Jetdüse zu bauen, die ein internes Wasserreservoir besitzt und während des Betriebs wiederbefüllt werden kann. Die Düse soll 
 
  
  Am Ende der Motivation noch kurz die Inhaltsangabe machen
  
  
%Ich habe vor eine Wasserdüse zu konstruieren und in Betrieb zu nehmen, die Wasserdimere produziert. Diese sollen in ein Remi als Target eingespeist werden und wir wollen dann wie erwähnt Icd triggern oder Proton transfer dynamics.

%Einleitung die den Leser abholt und langsam motiviert die Arbeit zu lesen. Interessante Punkte und Ziele vorstellen!
