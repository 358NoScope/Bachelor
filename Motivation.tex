
%%%%%%%%%%%%%%%%%%%%%%%%%%%%%%%%%%%%%%%%%%%%%%%%%%%%%%%%%%%%%%%%%%%%%%%%%
%%%%%%%%%%%%%%%%%%%%%%%%%%% M O T I V A T I O N %%%%%%%%%%%%%%%%%%%%%%%%%
%%%%%%%%%%%%%%%%%%%%%%%%%%%%%%%%%%%%%%%%%%%%%%%%%%%%%%%%%%%%%%%%%%%%%%%%%
  \chapter{Motivation} \label{sec:Motivation}
  
Wasser ist auf der Erde ein allgegenwärtiges Molekül, das den Grundstein des Lebens darstellt.
Obwohl es nur aus drei Atomen besteht, ein Sauerstoff- und zwei Wasserstoffatome, ist es noch nicht vollständig verstanden. Um Effekte wie die zahlreichen Anomalien von Wasser \cite{WAS} zu analysieren, ist es hilfreich nicht nur isolierte Moleküle in gasförmigen oder flüssigem Zustand zu untersuchen, sondern auch in Zwischenzuständen wie sie in Wasserclustern vorkommen. Die Eigenschaften von Wasserclustern gehen über die der einzelnen Wassermoleküle hinaus, aber unterscheiden sich auch von den Eigenschaften flüssigen Wassers. Bereits gewonnenes Wissen über die Vorgänge in Wasser finden Anwendung in Bereichen, die von medizinischer Strahlentherapie, bis hin zu Kernkraftwerktechnologien reichen.\\
Bei Experimenten über die Fragmentation von Wasserclustern wurde beobachtet, dass sich oft protonisierte Wassercluster bilden. Der kleinste Repräsentant davon ist das Oxonium-Ion ($\mathrm{H_3O^+}$). Grundlegend für dessen Entstehung ist der sogenannte Protonentransfer, über dessen Dynamik noch nicht viel bekannt ist. Es existiert allerdings eine Vielzahl von theoretischen Studien und Vorhersagen über die Protonentransferdynamik für verschiedene Größen von Wasserclustern. Dabei steigt die Genauigkeit der Berechnungen mit sinkender Komplexität der Cluster. Es bietet sich an den Protonentransfer in Wasser an den kleinsten Clustern, den Wasserdimeren, zu untersuchen. Die Ionisation von Wasserdimeren führt über einen Protonentransfer zur Entstehung von einem Oxoniumion und einem Hydroxylradikal (OH).
  \begin{equation}
  \textmd{H}_2\textmd{O}^+ + \textmd{H}_2\textmd{O} \rightarrow \textmd{H}_3\textmd{O}^+ + \textmd{O}\textmd{H} 
  \end{equation} 
Ein Reaktionsmikroskop ist ein präzises Impulsspektrometer, das geeignet ist diese Reaktion vollständig kinematisch zu untersuchen. Schnorr $et \ al.$ vom MPIK Heidelberg hat in einem aktuellen Proposal vorgeschlagen, diese Messung an FLASH in Hamburg durchzuführen. Die Zeitskala des Protonentransfers wird auf wenige 10 fs erwartet. Das DESY stellt den Freie-Elektronen-Laser in Hamburg (FLASH) zur Verfügung, der kurze XUV-Pulse mit einer Pulsdauer in der selben Größenordnung erzeugen kann und damit in der Lage ist, die Protonentransferdynamik aufzulösen. \\

Das Ziel dieser Bachelorarbeit ist es eine neue Cluster-Quelle für Flüssigkeiten zu konstruieren, die für das vorgestellte Experiment geeignet ist.\\

Das erste Kapitel arbeitet die physikalischen Grundlagen der Funktionsweise eines Reaktionsmikroskopes und der Entstehung von Clustern in einer Über-schallexpansion auf. \\
Die technische Umsetzung der Grundlagen wird im zweiten Kapitel behandelt. Die neu entwickelte Cluster-Quelle wird in Bezug auf die an sie gestellten Anforderungen und ihre Möglichkeiten vorgestellt. Des Weiteren wird erläutert wie technischen Komponenten im experimentellen Aufbau die Erzeugung eines kalten Targestjets ermöglichen.\\
Das letzte Kapitel befasst sich mit experimentellen Ergebnissen aus vergangenen Versuchen mit Wasserclustern und der Inbetriebnahme der neu gebauten Cluster-Quelle.