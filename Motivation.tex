
%%%%%%%%%%%%%%%%%%%%%%%%%%%%%%%%%%%%%%%%%%%%%%%%%%%%%%%%%%%%%%%%%%%%%%%%%
%%%%%%%%%%%%%%%%%%%%%%%%%%% M O T I V A T I O N %%%%%%%%%%%%%%%%%%%%%%%%%
%%%%%%%%%%%%%%%%%%%%%%%%%%%%%%%%%%%%%%%%%%%%%%%%%%%%%%%%%%%%%%%%%%%%%%%%%
  \chapter{Motivation} \label{sec:Motivation}
  
Wasser ist auf der Erde ein allgegenwärtiges Molekül, das den Grundstein des Lebens darstellt.
Obwohl es nur aus drei Atomen besteht, ein Sauerstoffatom und zwei Wasserstoffatome, ist es noch nicht vollständig verstanden. Um Effekte wie die zahlreichen Anomalien von Wasser \cite{WAS} zu analysieren, ist es hilfreich nicht nur isolierte Moleküle in gasförmigen oder flüssigem Zustand zu untersuchen, sondern auch in Zwischenzuständen wie sie in Wasserclustern vorkommen. Die Eigenschaften von Wasserclustern gehen über die der einzelnen Wassermoleküle hinaus, aber unterscheiden sich auch von den Eigenschaften flüssigen Wassers. Bereits gewonnenes Wissen über die Vorgänge in Wasser finden Anwendung in Bereichen, die von medizinischer Strahlentherapie, bis hin zu Kernkraftwerktechnologien reichen.\\
Bei Experimenten zur Fragmentation von Wasserclustern wurde beobachtet, dass sich häufig protonisierte Wassercluster bilden. Der kleinste Repräsentant davon ist das Oxoniumion ($\mathrm{H_3O^+}$). Grundlegend für dessen Entstehung ist der sogenannte Protonentransfer, über dessen Dynamik noch nicht viel bekannt ist. Es existiert allerdings eine Vielzahl von theoretischen Studien und Vorhersagen über die Protonentransferdynamik für verschiedene Größen von Wasserclustern. Dabei steigt die Genauigkeit der Berechnungen mit sinkender Komplexität der Cluster. Es bietet sich daher an den Protonentransfer in Wasser an den kleinsten Clustern, den Wasserdimeren, zu untersuchen. Die Ionisation von Wasserdimeren führt über einen Protonentransfer zur Entstehung von einem Oxoniumion und einem Hydroxylradikal (OH).
  \begin{equation}
  \textmd{H}_2\textmd{O}^+ + \textmd{H}_2\textmd{O} \rightarrow \textmd{H}_3\textmd{O}^+ + \textmd{O}\textmd{H} 
  \end{equation} 
Ein Reaktionsmikroskop ist ein präzises Impulsspektrometer, das geeignet ist diese Reaktion vollständig kinematisch zu untersuchen. Schnorr $et \ al.$ vom MPIK Heidelberg haben in einem aktuellen Proposal vorgeschlagen, dieses Experiment am Freie-Elektronen-Laser in Hamburg (FLASH) zeitaufgelöst durchzuführen. Die erwartete Dauer des Protonentransers liegt bei einigen 10 fs. FLASH kann kurze extrem-ultraviolette (XUV) Pulse mit einer Pulsdauer in der selben GRößenordnung erzeugen und mit einem XUV-XUV Pump-Probe-Experiment ist man in der Lage die Protonentransferdynamik aufzulösen. \\
\clearpage
Das Ziel dieser Bachelorarbeit ist es eine neue Cluster-Quelle für Flüssigkeiten zu bauen, die für das vorgestellte Experiment geeignet ist.\\

Das erste Kapitel befasst sich mit der Funktionsweise eines Reaktionsmikroskopes und der Entstehung von Clustern durch Überschallexpansion. \\
Die technische Umsetzung der Grundlagen wird im zweiten Kapitel behandelt. Die neu entwickelte Cluster-Quelle wird in Bezug auf die an sie gestellten Anforderungen und ihre Möglichkeiten vorgestellt. Desweiteren wird erläutert, wie technischen Komponenten im experimentellen Aufbau die Erzeugung eines kalten Targestjets ermöglichen.\\
Das letzte Kapitel befasst sich mit experimentellen Ergebnissen aus vergangenen Versuchen mit Wasserclustern und der Inbetriebnahme der neu gebauten Cluster-Quelle.