%%%%%%%%%%%%%%%%%%%%%%%%%%%%%%%%%%%%%%%%%%%%%%%%%%%%%%%%%%%%%%%%%%%%%%%%%
%%%%%%%%%%%%%%%%%%%%%%%%%%%%%% D E S I G N  %%%%%%%%%%%%%%%%%%%%%%%%%%%%%
%%%%%%%%%%%%%%%%%%%%%%%%%%%%%%%%%%%%%%%%%%%%%%%%%%%%%%%%%%%%%%%%%%%%%%%%%
\chapter{Technische Umsetzung} 

Der Hauptbestandteil dieser Arbeit bestand darin, eine verlässliche und praktische Wassercluster-Quelle zu konstruieren, die einen vakuumseitiges Reservoir besitzt und während des Betriebes wiederbefüllt werden kann, damit die bestmöglichen Vorraussetzungen für Experimente damit herrschen. (Hier etwas sinnvolles ausdenken + kleine Inhaltsübersicht)

\section{Düse}

Das Design wurde gewissermaßer durch die Arbeiten von Mucke \cite{mucke2011} und Wechselberger \cite{wechselberger2014} inspririert und durch eigene Überlegungen verfeinert.
%
%Wie und wieso wurde die Düse konstruiert wie sie ist. Konische Spitze wegen Gradient und Düsenöffnung so groß usw.
%
%Wie detailliert wird das? Aber schon alle Gedanken die wichtig waren erwähnen?

%Auf wichtige Details und Gründe bei der Konstruktion eingehen + Vergleich mit alter Bauweise!}


\section{Skimmer}

%Vielleicht erst hier die Rechnung mit Position?, aber auf jedenfall hier referenzieren

\section{Jetstufen}


%
%\section{Jet}
%
%
%
%
%
%Skimmer und Jetstages und Fokussierung des Strahls ins Remi $\mathrm{\rightarrow}$ Dump
%
%
%
