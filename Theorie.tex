%%%%%%%%%%%%%%%%%%%%%%%%%%%%%%%%%%%%%%%%%%%%%%%%%%%%%%%%%%%%%%%%%%%%%%%%%
%%%%%%%%%%%%%%%%%%%%%%%%%% Vorbereitung/Theorie %%%%%%%%%%%%%%%%%%%%%%%%%
%%%%%%%%%%%%%%%%%%%%%%%%%%%%%%%%%%%%%%%%%%%%%%%%%%%%%%%%%%%%%%%%%%%%%%%%%
\chapter{Experimentelle Realisierung} 

\section{Cluster}

Für die in der Motivation erwähnten Experimente benötigen wir eine Quelle, die ein Target aus Wasserdimeren zur Verfügung stellt. Wir haben uns entschieden, durch adiabatische Überschallexpansion solche Wassercluster zu realisieren. Die entsprechende Düse wurde von uns eigens zu diesem Zweck designt und wird im weiteren Verlauf der Arbeit noch eingehend besprochen[Referenz zum entsprechenden Kapitel].

%Missglückte Einleitung
%Was regt uns an Cluster zu untersuchen?
%Cluster sind Ansammlungen von einer bestimmten Anzahl an Atomen	oder Molekülen. Je nach Größe zeigen Sie Eigenschaften von Aggregatszuständen. Das ermöglicht es Übergänge von z.B gasförmig zu flüssig auf elementarer Ebene physikalisch zu Untersuchen.
% Auch andere Vorgänge wie die Gleichgewichtsreaktion von $\mathrm{H}_2\mathrm{O}+ \mathrm{H}_2\mathrm{O}^+ \rightarrow \mathrm{H}_3\mathrm{O}^+ + \mathrm{H}\mathrm{O}$ können in einem isolierten Rahmen untersucht werden,
% indem man nur die zwei Wassermoleküle in einem Cluster hat und reagieren lässt.\\

\subsection{Definition}
Als Cluster bezeichnet man Agglomerate aus Atomen oder Molekülen. Diese können bei kleinen Exemplaren 2 bis mehrere Hundert Teilchen enthalten und bei großen Clustern aus bis zu ca. $\mathrm{10^6}$ Teilchen bestehen. Je nach Größe besitzen Cluster verschiedene Eigenschaften und bilden damit eine Brücke zwischen der Atomphysik, der Molekülphysik und der Festkörperphysik.\\
Bei kleinen Clustern lassen sich noch alle Eigenschaften anhand der Atom- und Molekülphysik beschreiben. Ähnlich wie beim Atom besitzen kleine Cluster diskrete Energieniveaus.
Bis zu einer gewissen Größe, strukturieren sich die Cluster beim Dazukommen eines weiteren Atoms vollständig um und verändern auf diese Weise auch ihre physikalischen und chemischen Eigenschaften. Sobald das Cluster eine Größe von etwa 100 Atomen oder Molekülen erreicht, kann man allmählich die Anordnung in einer Gitterstruktur wie bei Festkörpern beobachten. Auch die anfangs diskreten Energieniveaus gehen langsam in ein kontinuierliches Energieband über[Referenzen zu Büchern einfügen].
Sogenannte Mikrokristalle sind Cluster die aus ca 1000 Atomen oder Molekülen bestehen. Die besitzen schon einige Eigenschaften von Festkörpern und bei Clustern in der Größe von etwa 50000 Konstituenten kann man von Festkörpern sprechen, da sie diesen in allen Eigenschaften ähneln. \\

Cluster können aus fast jeder Art von Atomen oder Molekülen erzeugt werden. Diese werden je nach ihren Bestandteilen und Bindungstypen in verschiedene Gruppen eingeteilt[HierReferenzWiki-Buch?]. \\

Da gibt es zum einen die Gruppe der metallischen Cluster. Diese z.B. $\mathrm{Al}_n$-Cluster bilden untereinander eine metallische Verbindung mit einem halbvollen Band delokalisierter Bindungselektronen. Die Bindungsenergien sind hierbei in einem Bereich von 0,5 - 3 eV. Metallische Cluster bestehen oft aus $\mathrm{n > 200}$ Atomen.

Dann gibt es die Gruppe der ionischen Cluster. Hier weisen die Bestandteile oft einen großen Unterschied in Elektronegativität auf und werden durch Coulombwechselwirkung zusammengehalten. Ihre Struktur ist meist kubisch wie wir das von $\mathrm{NaCl}_n$ kennen und die Bindungsenergien liegen zwischen 2 - 4 eV. 

Eine weitere bekannte Gruppe ist die der kovalenten Cluster, die vor allem durch C-Cluster wie Fullerene bekannt wurde. Diese besitzen eine ausgerichtete Bindung durch Elektronenpaare, mit einer mittleren Bindungsenergie von 1 - 4 eV. Bei diesem Typ liegt die charakteristische Clustergröße bei 30 $\mathrm{< n <}$ 80.

Weiterhin gibt es die Van-der-Waals-Cluster, die durch eine Dipol-Wechselwirkung zwischen Atomen und Molekülen verbunden werden. Sie besitzen nur eine geringe mittlere Bindungsenergie von 0,001 - 0,3 eV. Repräsentanten dieser Gruppe sind z.B. Edelgascluster wie $\mathrm{H}_2$-Cluster. Diese Cluster bestehen meist aus $\mathrm{n < 10}$ Atomen oder Molekülen .

Und zum Schluss gibt es noch die Gruppe, mit der wir uns in dieser Arbeit befassen werden, und zwar die Gruppe der Cluster mit Wasserstoffbindung. Die Dipol-Dipol-Anziehung hält diese Cluster mit einer mittleren Bindungsenergie von 0,15 - 0,5 eV zusammen. \\


\subsection{Überschallexpansion}

Die Überschallexpansion ist ein adiabatischer Prozess, bei dem Gas durch eine kleine Öffnung von einem Bereich mit hohem Druck, in einen Bereich mit niedrigem Druck strömt. Da die räumliche Dichte abnimmt, muss die Dichte im Impulsraum zunehmen, was einer Temperaturabnahme entspricht. Diese Temperaturabnahme ist sehr bemerkenswert, da sie es ermöglicht van-der-Waals-gebundene Moleküle wie z.B. $\mathrm{^4}\mathrm{He}_2$ zu produzieren und zu detektieren. Das $\mathrm{^4}\mathrm{He}_2$-Molekül ist ein sehr beeindruckendes Beispiel dafür, weil die He-He-Bindung nach quantenmechanischen Berechnungen die schwächste bekannte Bindung ist. Diese Moleküle wurden 1993 erstmals, via Überschallexpansion erzeugt und nachgewiesen \cite{Luo1993}, obwohl die Bindungsenergie  nur 1,176 mK, also etwa $\mathrm{10^{-4}}$ eV entspricht \cite{Lohr2007}. 

In der folgenden Erklärung von den Prozessen und Formeln die bei der Clusterbildung durch Überschallexpansion eine Rolle spielen orientiere ich mich im Wesentlichen am Kapitel 2 des Buches \enquote{Atomic and Molecular Beam Methods Vol 1}\cite{scoles1988} und der Zusammenfassung darüber in der Dissertation über \enquote{Ultraschnelle Dynamik in dotierten und reinen Wassercluster} von Jan Müller \cite{mul13}.

\begin{center}
\begin{minipage}{\linewidth}
\centering
\includegraphics[width=0.7\textwidth]{../expansion.png}%
\captionof{figure}{Übersicht der Überschallexpansion mit Schockwellenstruktur \citep{scoles1988}}  
 \label{fig:Machexpansion}
\end{minipage} 
\end{center} 

Wenn das Gas aus einem Reservoir mit Temperatur $t_0$ und Druck $p_0$ adiabatisch durch eine Düse mit Durchmesser d expandiert, folgt die Druckabnahme der Gesetzmäßigkeit

\begin{equation}
pV^{\gamma/(1-\gamma)}\ \mbox{mit}\ \gamma=c_p/c_v=(f+2)/f \ \mbox{bei} \ \Delta S=0.
\end{equation}

Der Exponent $\gamma$ zählt die Anzahl der Freiheitsgrade und kann wie in der Formel angegeben als Quotient aus der isobaren und isochoren Wärmekapazität experimentell bestimmt werden. Reines Wasser, wie es bei uns verwendet wird, hat bei Raumtemperatur 6 Freiheitsgrade [brauch ich hierfür Referenz? oder Fußnote mit Werten für id Gas oder Moleküle anhängen]. 
Da bei der Expansion ins Vakuum die Dichte und Temperatur des Gases sinken, muss aufgrund der Energieerhaltung, die Geschwindigkeit zunehmen. 
%Wenn man die Düsenöffnung als klein betrachtet (was bei ca 50$\mu$m berechtigt ist), darf man die Expansion eindimensional behandeln und damit kann man sich anschaulich vorstellen, dass die Beschleunigung in Richtung der Achse des Molekularstrahls erfolgen muss.
Nimmt man an dass die gesamte thermische Energie in kinetische Energie umgewandelt wird erhält man folgende Formel für die Endgeschwindigkeit.

\begin{equation}
v_\infty=\sqrt{2\int_{T_0}^{T_\infty \ll T_0} c_{p,mol}dT}= \sqrt{\frac{2R}{W}\left(\frac{\gamma}{\gamma-1}\right)T_0}
\end{equation}

Wobei der Zusammenhang für ideale Gase

\begin{equation}
c_{p,mol}= \left(\frac{\gamma}{\gamma-1}\right)\left(\frac{R}{W}\right)
\end{equation}

verwendet wurde. R steht hier f"ur die universelle Gaskonstante, W kennzeichent das Molekulargewicht des verwendeten Gases und $T_0$ steht für die Düsentemperatur.
Bei Gasgemischen muss man die, nach ihrem atomaren Anteil gewichteteten, Mittelwerte $\bar{c}_{p,mol}$ und $\bar{W}_{mol}$ verwenden.
Vergleicht man die Endgeschwindigkeit $v_{\infty}$ mit der Schallgeschwindigkeit für ideales Gas

\begin{equation}
c=\sqrt{\frac{R \gamma}{W}T}
\end{equation}

sieht man, dass $v_{\infty}$ größer ist als c, weil die Temperatur des expandierten Gases T sehr viel kleiner ist, als die Düsentemperatur $T_0$ und der andere Faktor $\sqrt{(2/\gamma -1)}=2,45$ (mit f=6 bei Wasser) den Unterschied nochmal verstärkt.\\
Die Machzahl 
\begin{equation}
M(\vec{r})= v(\vec{r})/c(p(\vec{r}))
\end{equation}

ist bei der Überschallexpansion eine wichtige Größe, die skalar in vielen thermodynamischen Rechnungen dazu eine Rolle spielt und vektoriell betrachtet das Strömungsfeld an jedem Ort $\vec{r}$ charakterisiert. 

Nach Austritt aus der Düse hat das Gas eine Machzahl M$>$1, was bedeutet, das Gas breitet sich mit Überschallgeschwindigkeit aus. Das hat zur Folge, das der Gasstrom zunächst unabhängig von jeglichen externen Randbedingungen ist. Dieser Effekt rührt daher, dass sich Information \enquote{nur} mit Schallgeschwindigkeit ausbreitet und das Fluid eben schneller (M$>$1) ist. Doch obwohl der Gasstrom nichts von Randbedingungen weiß, muss er sich an diese anpassen. Das wird nach kurzer Zeit von Schockwellen realisiert, die an begrenzenden Wänden oder ähnlichen \enquote{Randbedingungen} abprallen und auf ihrem Rückweg den Strom regulieren. Wie man an Bild \ref{fig:Machexpansion} erkennen kann gibt es mehrere Instanzen dieser Schockwellen, die nichtisentropische Gebiete sind und sich durch starke Dichte-, Temperatur-, Geschwindigkeits- und Druckgradienten auszeichnen. Durch diese Eigenschaften kann man die Schockwellen mit diversen Lichtstreutechniken sichtbar machen.

\begin{center}
\begin{minipage}{\linewidth}
\centering
\includegraphics[width=0.7\textwidth]{../shockwaves.png}%
\captionof{figure}{Druckverteilung bei einem Argon-Jet mit einer 20 µm breiten schlitzförmigen Düsenöffnung. Die gekennzeichnete Schockwelle in der Mitte begrenzt den isentropen Bereich, also die \enquote{zone of silence} \cite{Mou09} }
 \label{fig:Machexpansion}
\end{minipage} 
\end{center} 
 
Der von diesen Schockwellen eingegrenzte Bereich ist dennoch unbeeinflusst von den Randbedingungen und wird deswegen auch \enquote{zone of silence} genannt. 
Die Lage der Schockstrukturen hauptsächlich abhängig von dem Verhältnis von Düsendruck $p_0$ zu Hintergrunddruck $p_b$ der Kammer. Eine weitere wichtige Schockstruktur ist die sog. Mach-Scheibe (englisch \enquote{mach disc}), die normal zur Ausbreitungsrichtung des Gases liegt. Da Schockwellen enormen Druck auf die im Strom vorkommenden Cluster ausübt, muss der für Experimente verwendete Molekülstrahl noch vor der Mach-Scheibe entnommen werden. Diese Entnahme wird durch einen Skimmer realisiert, welcher den gewollten Teil des Molekülstrahls durchlässt und den restlichen Anteil von der Strahlachse wegreflektiert, doch auf diesen Teil werde ich im Kapitel [Kapitel über Jet/Skimmer] etwas genauer eingehen. 

Die Position der Mach-Scheibe lässt sich mittels

\begin{equation}
\frac{x_m}{d}=0.67 \sqrt{\left( \frac{p_0}{p_b}\right)}
\end{equation}

bestimmen, wobei $x_m$ die besagte Mach-Scheibenposition, ausgehend von der Düse und d die Größe der Düsenöffnung ist.
Um einen kleinen Vorausgriff zu wagen, habe ich hier schonmal grobe Werte unseres Experimentes eingesetzt. Mit d = 50 µm, $p_0 \approx$ 1000 mbar und $p_b \approx 10^{-3}$  mbar erhält man für\\ $x_m \approx 3,35$ cm. Ein sehr großer Wert, wenn man bedenkt, dass man üblicherweise wenige Millimeter nah an den Skimmer fährt und zudem der Wert in der Realität noch größer ist, weil ich hier für $p_0$ den Dampfdruck im Reservoir bei 100°C gewählt habe (man weiß noch nicht dass verdampft wird!) und nicht den an der Düse der garantiert größer sein wird.

 %Um das Strömungsfeld der Überschallexpansion zu charakterisieren muss man nicht nur die erwähnten Schockwelleneffekte betrachten, sondern sich auch die Form der Düse anschauen. Hier gibt es einen effektiven Düsendurchmesser $d_{eff}$ der z.B. bewirkt, dass konische Düsen bei geringerem Gasfluss, gleiche Gasdichten auf der Strahlachse und gleiche Clustergrößen produziert, wie eine zylindrische Düse mit gleichem $d_{eff}$ \cite{HOb72}. Den effektiven Düsendurchmesser
%Jetzt das mit eff Düsenöffnung

%Was ist mit Speedratio für Genauigkeit des Strahls? Ist durch relativbeweg doch eig schon dabei

\subsection{Clusterbildung}

Das zur Clusterproduktion verwendete reine Wasser wird bei Temperaturen um 100 °C und einem entsprechendem Dampfdruck von ca. 1 bar durch eine kurze konische Düse geleitet (Abb. \ref{fig:Duse}).

\begin{center}
\begin{minipage}{\linewidth}
\centering
\includegraphics[width=0.8\textwidth]{../duse.pdf}%
\captionof{figure}{Abmessungen der von uns benutzten Platin-Blenden von Plano als Düse. Als Düsenöffnung d haben wir 30 µm und 50 µm benutzt. D ist der Durchmeser des Plättchens und H die Höhe. Pfeile normal nachen!}
 \label{fig:Duse}
\end{minipage} 
\end{center} 

Nachdem es den engsten Querschnitt durchquert hat, kommt es zu einer starken Expansion und damit einhergehend zu einer starken Abkühlung des Gases (Joule-Thomson-Effekt). Durch das Abkühlen werden die Relativgeschwindigkeiten der Gasmoleküle sehr klein, sodass sich ein gerichteter Teilchenstrom einstellt. Wenn die thermische Energie der Moleküle unter die Bindungsenergie eines Dimers sinkt, kann es durch Dreikörperstöße zur Agglomeration zweier Moleküle kommen. Der dritte Körper ist bei diesem Prozess wichtig zur Bewahrung der Energie und Impulserhaltung. Die enstandenen Dimere dienen nun als Kondensationskeim, an den sich viele weitere Moleküle anlagern können. Allerdings, gibt jeder weiter sich anlagernde Moleküle die Bindungsenergie frei, durch die sich das Cluster aufheizt. Damit wird die Stabilität natürlich eingeschränkt, was dazu führt dass einige Moleküle wieder abdampfen. Wenn ein hoher Gasdruck herrscht, sprich eine große Teilchendichte existiert, gibt es mehr Teilchen die durch Stöße mit dem Cluster die überschüssige Energie abtransportieren können und somit die Clusterbildung unterstützen.
Die Größe der Cluster die bei einer Überschallexpansion lässt sich durch eine Boltzmannverteilung beschreiben, deren große Breite proportional zur mittleren Clustergröße $\left\langle N \right\rangle$ ist. 

Um auf die Clusterverteilungen in idealen Gasen schließen, benutzt man den empirischen Skalenparameter $\Gamma$ von Hagena \cite{hagena1987}:

\begin{equation}
\Gamma = n_0\ d^q\ T_0^{\alpha} \quad (0,5 < q \leq 1),\quad \alpha := sp - f/2
\end{equation}

Für axialsymmetrische Flüsse gilt $s = (f-2)/4$ und q ist ein Parameter der experimentell bestimmt werden muss.

Um $\Gamma$ unabhängig von den Einheiten zu machen wird der reduzierte Skalenparameter

\begin{equation}
\Gamma^* = \Gamma / (r_{ch}^{q-3}\ T_{ch}^{\alpha})
\end{equation}

eingeführt mit $\alpha= 1,5- 0,25q$. Dabei sind

\begin{equation}
r_{ch} = \frac{m_{atom}}{\rho ^{1/3}}\quad \textmd{und}\quad T_{ch}= \frac{ \Delta h_{atom,0}}{k_B}
\end{equation}

$r_{ch}$ bringt hierbei die Eigenschaften des Gases in die Gleichung ein, wobei m die Atommasse und $\rho$ die Festkörperdichte ist. $T_{ch}$ ist die charakteristische Temperatur mit der Sublimationsenthalpie $\Delta h_{atom,0}$ bei 0 K  und der Boltzmannkonstante $k_B$. Die mittlere Clustergröße ergibt sich nach Hagena als

\begin{equation}
\left\langle N \right\rangle  = D \left( \frac{\Gamma*}{1000}\right)^a \:,
\end{equation}

wobei a und D experimentell ermittelt werden müssen.

%Brauch ich jetzt die Verbesserungen von Buck und Krohne?

%Ich könnte hier jetzt mit Buck und Krohne vergleichen oder direkt das Beispiel von Jan Müller bringen







\newpage

\section{Jet}

Skimmer und Jetstages und Fokussierung des Strahls ins Remi $\mathrm{\rightarrow}$ Dump

%Vergleichstext 

\section{(REMI?) eher Quadrupol}

Grundlegendes Prinzip des REMIs, Ionisation, Detektoren, MCP, DelayLineAnode

%Gute Beschreibungen bei Kirsten und Lutz

