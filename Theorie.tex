%%%%%%%%%%%%%%%%%%%%%%%%%%%%%%%%%%%%%%%%%%%%%%%%%%%%%%%%%%%%%%%%%%%%%%%%%
%%%%%%%%%%%%%%%%%%%%%%%%%% Vorbereitung/Theorie %%%%%%%%%%%%%%%%%%%%%%%%%
%%%%%%%%%%%%%%%%%%%%%%%%%%%%%%%%%%%%%%%%%%%%%%%%%%%%%%%%%%%%%%%%%%%%%%%%%
\chapter{Experimentelle Realisierung} 

\section{Cluster}

Für die in der Motivation erwähnten Experimente benötigen wir eine Quelle, die ein Target aus Wasserdimeren zur Verfügung stellt. Wir haben uns entschieden, durch adiabatische Überschallexpansion solche Wassercluster zu realisieren. Die entsprechende Düse wurde von uns eigens zu diesem Zweck designt und wird im weiteren Verlauf der Arbeit noch eingehend besprochen[Referenz zum entsprechenden Kapitel].

\subsection{Definition}
Als Cluster bezeichnet man Agglomerate aus Atomen oder Molekülen. Diese können bei kleinen Exemplaren 2 bis mehrere Hundert Teilchen enthalten und bei großen Clustern aus bis zu ca. $\mathrm{10^6}$ Teilchen bestehen. Je nach Größe besitzen Cluster verschiedene Eigenschaften und bilden damit eine Brücke zwischen der Atomphysik, der Molekülphysik und der Festkörperphysik.\\
Bei kleinen Clustern lassen sich noch alle Eigenschaften anhand der Atom- und Molekülphysik beschreiben. Ähnlich wie beim Atom besitzen kleine Cluster diskrete Energieniveaus.
Bis zu einer gewissen Größe, strukturieren sich die Cluster beim Dazukommen eines weiteren Atoms vollständig um und verändern auf diese Weise auch ihre physikalischen und chemischen Eigenschaften. Sobald das Cluster eine Größe von etwa 100 Atomen oder Molekülen erreicht, kann man allmählich die Anordnung in einer Gitterstruktur wie bei Festkörpern beobachten. Auch die anfangs diskreten Energieniveaus gehen langsam in ein kontinuierliches Energieband über[Referenzen zu Büchern einfügen].
Sogenannte Mikrokristalle sind Cluster die aus ca 1000 Atomen oder Molekülen bestehen. Die besitzen schon einige Eigenschaften von Festkörpern und bei Clustern in der Größe von etwa 50000 Konstituenten kann man von Festkörpern sprechen, da sie diesen in allen Eigenschaften ähneln. \\

Cluster können aus fast jeder Art von Atomen oder Molekülen erzeugt werden. Diese werden je nach ihren Bestandteilen und Bindungstypen in verschiedene Gruppen eingeteilt[HierReferenzWiki-Buch?]. \\

Da gibt es zum einen die Gruppe der metallischen Cluster. Diese z.B. $\mathrm{Al}_n$-Cluster bilden untereinander eine metallische Verbindung mit einem halbvollen Band delokalisierter Bindungselektronen. Die Bindungsenergien sind hierbei in einem Bereich von 0,5 - 3 eV. Metallische Cluster bestehen oft aus $\mathrm{n > 200}$ Atomen.

Dann gibt es die Gruppe der ionischen Cluster. Hier weisen die Bestandteile oft einen großen Unterschied in Elektronegativität auf und werden durch Coulombwechselwirkung zusammengehalten. Ihre Struktur ist meist kubisch wie wir das von $\mathrm{NaCl}_n$ kennen und die Bindungsenergien liegen zwischen 2 - 4 eV. 

Eine weitere bekannte Gruppe ist die der kovalenten Cluster, die vor allem durch C-Cluster wie Fullerene bekannt wurde. Diese besitzen eine ausgerichtete Bindung durch Elektronenpaare, mit einer mittleren Bindungsenergie von 1 - 4 eV. Bei diesem Typ liegt die charakteristische Clustergröße bei 30 $\mathrm{< n <}$ 80.

Weiterhin gibt es die Van-der-Waals-Cluster, die durch eine Dipol-Wechselwirkung zwischen Atomen und Molekülen verbunden werden. Sie besitzen nur eine geringe mittlere Bindungsenergie von 0,001 - 0,3 eV. Repräsentanten dieser Gruppe sind z.B. Edelgascluster wie $\mathrm{H}_2$-Cluster. Diese Cluster bestehen meist aus $\mathrm{n < 10}$ Atomen oder Molekülen .

Und zum Schluss gibt es noch die Gruppe, mit der wir uns in dieser Arbeit befassen werden, und zwar die Gruppe der Cluster mit Wasserstoffbindung. Die Dipol-Dipol-Anziehung hält diese Cluster mit einer mittleren Bindungsenergie von 0,15 - 0,5 eV zusammen. \\

\textit{Doch was ist nun der Grund warum wir Cluster als Target nehmen?
%Ich denke man brauch dafür keine so ausgefeilte Düse. einfach ein reservoir in dem geheizt wird und eine relativ Gr0ße Öffnung oder?
Bei den von uns angestrebten Versuchen wollen wir 2 Wassermoleküle isoliert betrachten. Dazu eignen sich Cluster ideal, denn durch die Einphoton-Ionisation, die wir auf das Target anwenden, wählen wir gezielt ein Cluster aus dem Jet-Target aus und beobachten durch das Remi wie die darauffolgende Reaktion abläuft. Was das Remi ist und wie es funktioniert wird im Kapitel [Referenz] kurz zusammengefasst.}

\subsection{Überschallexpansion}

Die Überschallexpansion ist ein adiabatischer Prozess, bei dem Gas durch eine kleine Öffnung von einem Bereich mit hohem Druck, in einen Bereich mit niedrigem Druck strömt. Da die räumliche Dichte abnimmt, muss die Dichte im Impulsraum zunehmen, was einer Temperaturabnahme entspricht. Diese Temperaturabnahme ist sehr bemerkenswert, da sie es ermöglich van-der-Waals-gebundene Moleküle wie z.B. $\mathrm{^4}\mathrm{He}_2$ zu produzieren und anschließend zu detektieren bzw. zu untersuchen \cite{Liu89}.


%Missglückte Einleitung
%Was regt uns an Cluster zu untersuchen?
%Cluster sind Ansammlungen von einer bestimmten Anzahl an Atomen	oder Molekülen. Je nach Größe zeigen Sie Eigenschaften von Aggregatszuständen. Das ermöglicht es Übergänge von z.B gasförmig zu flüssig auf elementarer Ebene physikalisch zu Untersuchen.
% Auch andere Vorgänge wie die Gleichgewichtsreaktion von $\mathrm{H}_2\mathrm{O}+ \mathrm{H}_2\mathrm{O}^+ \rightarrow \mathrm{H}_3\mathrm{O}^+ + \mathrm{H}\mathrm{O}$ können in einem isolierten Rahmen untersucht werden,
% indem man nur die zwei Wassermoleküle in einem Cluster hat und reagieren lässt.\\







%Supersonicexpansion, Machkegel, Die ganzen Formeln dazu

%Vergleichstext 

\section{Jet}

Skimmer und Jetstages und Fokussierung des Strahls ins Remi $\mathrm{\rightarrow}$ Dump

%Vergleichstext 

\section{(REMI?) eher Quadrupol}

Grundlegendes Prinzip des REMIs, Ionisation, Detektoren, MCP, DelayLineAnode

%Gute Beschreibungen bei Kirsten und Lutz

