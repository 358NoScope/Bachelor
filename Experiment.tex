%%%%%%%%%%%%%%%%%%%%%%%%%%%%%%%%%%%%%%%%%%%%%%%%%%%%%%%%%%%%%%%%%%%%%%%%%
%%%%%%%%%%%%%%%%%%%%%%%%%%%%%%% Experiment  %%%%%%%%%%%%%%%%%%%%%%%%%%%%%
%%%%%%%%%%%%%%%%%%%%%%%%%%%%%%%%%%%%%%%%%%%%%%%%%%%%%%%%%%%%%%%%%%%%%%%%%
\chapter{Experiment} 
\section{Messungen an FLASH}
\subsection{Alte Wasserdüse}

Im Juni 2014 wurden von der Gruppe um Moshammer vom MPIK Heidelberg bereits Messungen mit Wasserclustern an FLASH in Hamburg durchgeführt. Der Aufbau der verwendeten Wasserclusterquelle ist in Abbildung \ref{fig:iodquelle} zu sehen. Diese Quelle ist dafür ausgelegt Flüssigkeiten, als auch Feststoffe wie Iod im Reservoir 4 zu verdampfen und mit einem Trägergas aus Reservoir 1 in das Reaktionsmikroskop zu führen. Bei den Experimenten mit Wasser wurde ohne Trägergas gearbeitet. Das Wasser wurde im beheizbaren Reservoir 4 auf 100 °C erhitzt und der entstehende Wasserdampf in der ersten Jetstufe durch eine Düse ins Vakuum expandiert.
Zur Analyse der gemessenen Daten ist ein speziell auf Reaktionsmikroskope angepasster Analysecode in das Programm Go4 (GSI Object Oriented On-line Off-line system) eingebettet. Die Flugzeitspektren und Ortsbilder in diesem Kapitel zeigen die Messdaten vom Juni 2014. 

\begin{center}
\begin{minipage}{\linewidth}
\centering
\includegraphics[width=0.8\textwidth]{../iodquelle.png}%
\captionof{figure}{Schematische Darstellung der Quelle. 1: Reservoir des Trägergases. 2: Druckminderer. 3 und 6: Dreiwegehahn. 4: Reservoir zur Verdampfung von Feststoffen. 5: Pumpe zum Spülen von Feststoffreservoir und Trägergaszuleitung. 7: Filter. 8: Erste Jetstufe des REMI. Die Komponenten 4,6,7 und deren Zuleitungen sind mit Heizband umwickelt (grau angedeutet).}
 \label{fig:iodquelle}
\end{minipage} 
\end{center} 

\subsection{Flugzeitspektren}

Flugzeitspektren (TOF-Spektren\footnote{Häufig wird die ans Englische angelehnte Abkürzung \textbf{T}ime-\textbf{O}f-\textbf{F}light-Spektrum benutzt}) tregen die Häufigkeit detektierter Ionen gegen ihre Flugezit im Spektrometer auf. Die Flugzeit eines geladenen Teilchens mit Masse $m$ und Ladung $Q$ ist abhängig von der Spektrometerlänge $L_S$ und der angelegten Spannung $U$.
\begin{equation} \label{eq:Beschleunigung}
L_S = \frac{1}{2}at^2\qquad \rightarrow \qquad t = \sqrt{\frac{2L_S}{a}} \: ,
\end{equation} 
wobei die Beschleunigung im elektrischen Feld gegeben ist durch
\begin{equation}
F = m \cdot a = \frac{QU}{L_S}
\end{equation}
Setzt man $a$ in Gleichung \ref{eq:Beschleunigung} ein erhält man für die Flugzeit den Zusammenhang 
\begin{equation} \label{eq:Flugzeit}
t = L_S \sqrt{\frac{2m}{QU}}\: \sim \: \sqrt{\frac{m}{Q}} \: .
\end{equation}
Die Flugzeit der Ionen ist proportional zu der Wurzel aus dem Verhältnis von Masse zu Ladung. Kann ein markanter Peak aus dem Spektrum einem Ion zugeordnet werden, so ist aufgrund der Proportionalität in Gleichung \ref{eq:Flugzeit} die Berechnung der Flugzeiten unbekannter Peaks unter experimentellen Bedingungen möglich. Durch Auswertung des Flugzeitspektrums kann man genau untersuchen, welche Teilchen sich im Reaktionsvolumen befinden. Betrachtet man das Spektrum aller detektierten Ionen, kann man zwischen scharfen Peaks aus dem Gasjet und einem breiten Untergrund von Restgasionen unterscheiden. Ionen die aus dem Gasjet entstehen haben aufgrund der Überschallexpansion eine geringe Temperatur und damit eine schmale Geschwindigkeitsverteilung (siehe Kapitel \ref{sec:uberschallexp}). Ionisiert der FEL\footnote{\textbf{F}reie \textbf{E}lektronen \textbf{L}aser} Restgas in der Spektrometerkammer, weisen die Restgasionen breitere Peaks im Spektrum auf, da diese sich auf Raumtemperatur befinden und somit eine breite Geschwindigkeitsverteilung besitzen. In Abbildung \ref{fig:tofiall} kann man in linearer Darstellung vier deutliche Peaks erkennen. In logarithmischer Darstellung erkennt man neben den vier markanten Peaks noch viele breite und schwach ausgeprägte Peaks im Untergrund. Diese werden hauptsächlich von Restgasatomen verursacht.

\begin{center}
\begin{minipage}{\linewidth}
\centering
\includegraphics[width=1\textwidth]{../TofiAllLin.pdf} \\ ~\\~ \\
\includegraphics[width=1\textwidth]{../TofiAllLog.pdf}%
\captionof{figure}{Häufigkeiten der gemessenen Flugzeiten mit einem Wasserjet in linearer (oben) und logarithmischer (unten) Darstellung. In der logarithmischen Darstellung sind auch schwache Peaks gut zu erkennen. Der dominierende Peak 4 wurde Wasser (H$_2$O$^+$) mit $m$ = 18 u zugeordnet. Daraus folgt: Peak 3: OH$^+$, Peak 2: H$_2^+$, Peak 1: H$^+$.  }
 \label{fig:tofiall}
\end{minipage} 
\end{center} 
%Können aber markante Peaks aus dem Spektrum bestimmten Ionen zugeordnet werden, so ermöglicht die Proportionalität in Gleichung \ref{eq:Flugzeit} eine Umrechnung der Flugzeiten unbekannter Peaks \textbf{in das $m/Q$-Verhältnis unter experimentellen Bedingungen.} 
%, oder weisen in wenigen Fällen auf Vorgänge wie ICD und anschließende Coulombexplosion hin (Verbreiterung nach unten hin bei Peak 3 in Abbildung \ref{fig:tofiall})
\subsection{Ortsbild}

Trägt man alle Auftrefforte der detektierten Ionen in einem zweidimensionalen Histogramm auf (siehe Abbildung \ref{fig:Detektorbild}), erkennt man deutlich die Spur des FELs. Alle Teilchen aus dem Gasjet haben näherungsweise die gleiche Geschwindigkeit. Werden diese Teilchen am Laserfokuspunkt im Zentrum des Diagramms ionisiert treffen sie aufgrund ihrer gerichteten Anfangsgeschwindigkeit mit einem Offset in Jet-Richtung (hier negative x-Richtung) auf den Detektor. 
\begin{center}
\begin{minipage}{\linewidth}
\centering
\includegraphics[width=1\textwidth]{../Detektorbild.pdf}%
\captionof{figure}{Detektorbild aus einer Messung mit H$_2$O. Die vertikale Linie in y-Richtung stammt von dem vom FEL ionisierten Restgas. Die hohen Zählraten in dem gepunktet markierten Bereich sind auf Ionen aus dem Düsenstrahl zurückzuführen. Moleküle die durch eine Coulomb Explosion in zwei geladene Ionen dissoziieren erhalten dabei große Impulse in alle Raumrichtungen. Aus diesem Grund sind sie innerhalb des gestrichelten Kreises auf dem Detektor verteilt.}
 \label{fig:Detektorbild}
\end{minipage} 
\end{center} 

\subsection{Auswertung}

Untersucht man das Flugzeitspektrum aus dem Bereich der Jet-Ionen (siehe Abbildung \ref{fig:Conditions}), kann man nachvollziehen welche Ionen aus dem Teilchenstrahl entstehen. Um dabei den Untergrund von Messwerten trennen zu können setzt man die Ortsbedingung der Analyse auf einen Ort an dem ähnlicher Untergrund herrscht (siehe Abb \ref{fig:Conditions}, rechts). Da der FEL unabhängig von der Raumrichtung ionisiert, ist die Spiegelung der Ortsbedingung der Jetionen an der FEL-Achse ideal um den Untergrund zu messen.
\begin{center}
\begin{minipage}{\linewidth}
\centering
\includegraphics[width=0.5\textwidth, height=6.85cm]{../CONJET0.pdf}%
\includegraphics[width=0.5\textwidth]{../UntergrundJetCon.pdf}%
\captionof{figure}{Zur Untersuchung der Ionen aus dem Jet wurden nur Zählraten innerhalb der linken Ortsbedingung berücksichtigt. Der Untergrund wurde aus der rechten Ortsbedingung ermittelt.}
 \label{fig:Conditions}
\end{minipage} 
\end{center} 

\begin{center}
\begin{minipage}{\linewidth}
\centering
\includegraphics[width=0.91\textwidth]{../JetmitUntergrund.pdf}%
\captionof{figure}{Flugzeitspektrum der Ionen aus dem Jet.  Die eingetragenen Ionen wurden relativ zum dominierenden Wasserpeak bestimmt. Der Untergrund wurde auf $N_2^+$ und $H_2^+$ skaliert.}
 \label{fig:TOFJet}
\end{minipage} 
\end{center} 
Nähert man die Peaks mit einer Gaußkurve und integriert über diese, erhält man die Zählraten der einzelnen Peaks. Ist das Gesamtintegral des TOF-Spektrums bekannt, kann man daraus die Anteile der verschiedenen Ionen an der Gesamtverteilung berechnen (siehe Tabelle \ref{tab:Antile}). Die Peaks von Hintergrundgasen wie H$_2^+$ waren im Flugzeitspektrum mit Ortsbedingung auf dem Untergrund im unskalierten Zustand teilweise höher, als deren Peaks in dem Flugzeitspektrum mit Ortsbedingung auf den Jetionen. Das ist ein Indiz dafür, dass der Wasserpeak in dem TOF mit Ortsbedingung auf dem Jet in Sättigung gegangen ist und die Gesamtzahl der gemessenen Ionen somit nicht stimmt. Es wurde deswegen davon abgesehen die Zählraten der Untergrundpeaks abzuziehen, da man ohne richtigen Korrekturfaktor die Messwerte verfälscht hätte. 
%
\begin{center}
%\begin{minipage}{\linewidth}
\begin{table}[H]
\resizebox{\linewidth}{!}{
\begin{tabular}{c c c c c c c} \toprule
		 $TOF$ [ns] & $\sigma \ TOF$ [ns] & Ion & Masse $m$ [u] & Integral& Verh. zu Wasser& Anteil [\%]\\\midrule
            &   & Untergrund & & 1129882& 5,01 & 12,27\\
       4836 & 7  & N$_2^+$ & 28 & 1656& 3666,3 & 0,02 \\
       1300 & 18 & H$_2^+$ & 2 & 10608 & 907,0 & 0,07 \\
       1792 & 46 & He$^+$ / (H$_2$O)$_3^+$ & 4 / 55 & 11079& 522,8 & 0,12 \\
       500 & 7 & (H$_2$O)$_2^+$ & 36 & 19545 & 297,7 & 0,21 \\
       4089 & 7 & Ne$^+$ & 20 & 25123 & 225,3 & 0,27 \\
       575 & 49 & (H$_2$O)H$_3$O$^+$ & 37 & 108612& 55,0 & 1,12 \\
       923 & 20 & H$^+$ & 1 & 314055 & 25,9 & 2,37 \\
       3988 & 20 & (H$_3$O)$^+$ & 19 & 206706 & 25,6& 2,40 \\
       3770 & 15 & (OH)$^+$ & 17 & 1547230 & 3,6 & 16,95 \\
       3880 & 4 & (H$_2$O)$^+$ & 18 & 5784400 & 1 & 61,49 \\\bottomrule  
	  Gesamteinträge: & &&& 9205120& & 100 \\\bottomrule
\end{tabular}}
 \caption{Auswertung der Ionenanteile aus Abbildung \ref{fig:TOFJet}. Die Zeile \enquote{Untergrund} fasst schwach ausgeprägte und/oder sehr breite Peaks (N$^+$, O$^+$, O$_2^+$ etc.) im Diagramm mit Ortsbedingung auf dem Jet zusammen.} \label{tab:Antile}
\end{table}
%\end{minipage}
\end{center} 
Interessant sind die deutlichen Anteile von He$^+$ und Ne$^+$, die nicht aus dem Waserjet stammen können. Es hat sich herausgestellt, dass vor der Wassermessung Experimente mit Helium und Neon ausgeführt wurden. Da beide Peaks aber nicht im Untergrund erkennbar sind, müssen diese aus der Jet-Quelle gespült worden sein. Ionisiertes Wasser ist der bei weitem dominierende Anteil des Gasjets. Die nächstgrößten Bestandeile sind Zerfallsprodukte von Wasser, wie (H$_3$O)$^+$, oder (OH)$^+$. Wie man an der Zusammensetzung des Spektrums erkennen kann, ist die Wasserdimerausbeute bei diesem Experiment mit 0,21 \% sehr gering. Das Verhältnis von Wasser zu Wasserdimeren beträgt 297,7. Diese Wasserclusterquelle würde lange Messzeiten erfordern, um für die geplanten Wasserexperimente genügend Statistik zu sammeln (siehe Kapitel \ref{sec:Motivation}).
\newpage

\section{Testmessungen der neuen Wasserdüse}

Im Kapitel \ref{sec:Cluster} wurden die physikalischen Grundlagen zusammengefasst, die nun verwendet werden, um die in Kapitel \ref{sec:Quelle} behandelte Cluster-Quelle aufzubauen.
Bevor die Jetdüse für erste Tests in Betrieb genommen werden kann, müssen Heizdrahtwicklungen und Thermofühler angebracht werden. Die Stromleitungen der Thermofühler, Heizdrähte und der Heizpatrone werden durch mehrpolige \linebreak Durchführungen im Manipulatorflansch, aus dem Vakuum nach Außen übersetzt. 

\begin{center}
\begin{minipage}{\linewidth}
\centering
\includegraphics[width=1\textwidth]{../DuseBetrieb.pdf}%
\captionof{figure}{Einsatzbereite Jetdüse im Manipulator. An Spitze und Reservoir sind Heizdrahtwicklungen in roter Isolierung zu erkennen.}
 \label{fig:DuseBetrieb}
\end{minipage} 
\end{center} 

\subsection{Ausrichtung der Jetdüse} \label{sec:Ausrichtung der Jetduse}

Der experimentelle Aufbau, in dem die Jetdüse getestet wird besitzt insgesamt acht Druckstufen (siehe Abbildung \ref{fig:Jetstageaufbau}). In der ersten Jetstufe kann der Abstand in Austrittsrichtung (Z) und die Lage (X,Y) zwischen Düse und Skimmer, durch einen Manipulator während des Betriebes von außen verändert werden. Der Manipulator hat 50 mm Hub in Z-Richtung und  Zunächst wird die Düse mit 2 bar Argon betrieben, da dieses unkomplizierter zu handhaben ist als Wasser und weniger Gefahr für die Turbomolekularpumpen am Aufbau darstellt. 

\begin{center}
\begin{minipage}{\linewidth}
\centering
\includegraphics[width=1\textwidth]{../Teststand.pdf}%
\captionof{figure}{\small Teststand der Düse in Heidelberg}
 \label{fig:Teststand2}
\end{minipage} 
\end{center} 


Für optimale experimentelle Bedingungen wird der Kernstrahl der Expansion mit dem Skimmer extrahiert. Da sich der Jetdump in der Verlängerung der Symmetrieachse des Skimmers befindet, kann man an dem dort gemessenen Druck feststellen, wie gut der Gasstrahl den Teststand durchquert.
Sobald der Druck im Jetdump ein Maximum erreicht ist die optimale Einstellung für die Position der Düse gefunden. Zunächst ist zu beachten, dass ein kalter Teilchenstrahl vor der Machscheibe, also in der zone of silence entnommen werden muss.
Bei einem Argondruck von \linebreak p$_0$ = 2 bar stellt sich in der ersten Jetstufe ein Hintergrunddruck von \linebreak $p_b$ = 3,7 $\cdot$ 10$^{-3}$ mbar ein. Nach Formel \ref{eq:Machscheibe} beträgt die Entfernung $x_m$ zwischen Machscheibe und der Düse, je nach verwendetem Düsendurchmesser

\begin{equation}
   \begin{split}
   \frac{x_m}{d} & =0,67 \sqrt{\left(\frac{p_0}{p_b}\right)} = 0,67 \sqrt{\left(\frac{2000 \ \textmd{mbar}}{3,7 \cdot 10^{-3}\ \textmd{mbar}}\right)}\\ 
   \Longrightarrow x_m  & =  
   \left\{ %
   \begin{array}{l}
   0,0246\ \textmd{m} \textmd{ für d}= 50 \textmd{ µm} \\ 
   0,146\ \textmd{m} \textmd{ für d}= 30 \textmd{ µm}
   \end{array}
   \right. .
   \end{split}
\end{equation}


Solange die Z-Einstellung am Manipulator $x_m$ unterschreitet, richtet sich die genaue Position nach der gewünschten Strahlintensität in der Hauptkammer. Es ist allerdings günstig die Expansion so lange nicht mit Kanten zu stören, wie noch eine Kühlung der Moleküle durch Stöße erfolgt. Der parallele Verlauf der Drücke in den einzelnen Druckstufen bei logarithmischer Skala (siehe Abb. \ref{fig:Z}), zeugt von einer exponentiellen Abhängigheit der Drücke vom Abstand zwischen Düse und Skimmer. Setzt man die Düse auf eine feste Z-Position, hier 15,2 mm vor dem Skimmer bei einer Düsenöffnung von 50 µm, kann man die Druckverläufe für Düsenpositionen neben dem Optimum vermessen (siehe Abb. \ref{fig:Skimmerposi}/\ref{fig:Dump}).
\begin{center}
	\begin{minipage}{\linewidth}
		\centering
		\includegraphics[width=0.9\textwidth]{../Z-Pos.pdf}%
		\captionof{figure}{Der Graph zeigt die Druckverläufe der einzelnen Jetstufen in Abhängigkeit des Abstandes zwischen Düse und Skimmer in Z-Richtung mit logarithmischer Druckskala.}
		\label{fig:Z}
	\end{minipage} 
\end{center} 


\begin{center}
	\begin{minipage}{\linewidth}
		\centering
		\includegraphics[width=0.8\textwidth]{../Dump.pdf}%
		\captionof{figure}{Der Graph zeigt den Druckverlauf am Dump in Abhängigkeit der Düsenposition in X- und Y-Richtung bei einem festen Abstand zum Skimmer von 15,2 mm. Die Nullposition wurde am Maximum ausgerichtet.}
		\label{fig:Dump}
	\end{minipage} 
\end{center} 

\begin{center}
	\begin{minipage}{\linewidth}
		%\centering
		\includegraphics[width=1\textwidth]{../Jetx-Pos.pdf} \\ \newline
		\includegraphics[width=1\textwidth]{../Jety-Pos.pdf}%
		\captionof{figure}{Die Graphen zeigen den Druckverlauf an den verschiedenen Druckstufen in Abhängigkeit der transversalen Düsenposition in X- und Y-Richtung bei einem festen Abstand zum Skimmer von 15,2 mm.}
		\label{fig:Skimmerposi}
	\end{minipage} 
\end{center} 
\newpage

\subsection{Heiztest}

Die neue Jetdüse muss zur Wasserclusterproduktion mit stabilen Temperaturen betrieben werden können. Um ein Maß für die benötigen Zeitskalen beim Heizen und Abkühlen zu erhalten, wurde die Düse ohne Gaslast oder Wasser beheizt und dabei der Temperaturverlauf an Spitze und Reservoir aufgezeichnet (siehe Abbildung \ref{fig:Tempverlaufganz}).

\begin{center}
\begin{minipage}{\linewidth}
\centering
\includegraphics[width=1\textwidth]{../Beides.pdf}%
\captionof{figure}{Der Graph zeigt den Temperaturverlauf an Spitze und Reservoir der Jetdüse zu moderaten Heizströmen. Im ersten Abschnitt des Graphs wurde der Temperaturverlauf bei gleichen Heizströmen untersucht. Der zweite Abschnitt zeigt die Entwicklung der Temperatur bei gleichem Heizstrom am Reservoir, aber höherem Heizstrom an der Düse. Die Plateautemperatur beträgt bei dieser Konfiguration 121 °C an der Spitze und 112 °C am Reservoir, was geeignete Temperaturen für den Einsatz mit Wasser sind. Der dritte Abschnitt löst die Abkühlung der Düse auf.}
 \label{fig:Tempverlaufganz}
\end{minipage} 
\end{center} 

Der Heiz- und Abkühlvorgang wird gut durch beschränktes Wachstum beschrieben. Bei beiden Vorgängen ist nach drei Stunden etwa 90 \% der Endtemperatur erreicht. Bei Heizbetrieb stellt sich ein Plateau ein, sobald sich die zugeführte Wärme der Heizwicklung und die abgestrahlte Wärme der Jetdüse im Gleichgewicht befinden. Bis dieser Zustand eintritt dauert es mehrere Stunden, da die Düse aus Edelstahl 1.4301 besteht. Dieses gehört mit einem niedrigen Wärmeleitkoeffizienten von $\alpha = 15\ \frac{\textmd{W}}{\textmd{m}\cdot \textmd{K}}$ zu den am schlechtesten wärmeleitenden Metallen. Um die Wartezeit zu beschleunigen müssen die Heizströme reguliert werden. Heizt man mit einem Strom von I$_{Spitze} = 0,82$ A und I$_{Reservoir} = 0,79$ A, misst man bereits nach 15 Minuten 100 °C am Reservoir und 90 °C an der Spitze (siehe Abb. \ref{fig:Tempverlauf8A}). Wechselt man nun auf niedrigere Heizströme von I$_{Spitze} = 0,45$ A und I$_{Reservoir} = 0,41$ A erreicht man innerhalb einer Stunde über 98 \% der Plateautemperatur aus Abbildung \ref{fig:Tempverlaufganz}. Allerdings ist hierbei zu beachten, dass mit eingefülltem Wasser mehr Masse erhitzt werden muss. Dies kann aber durch einen leicht höheren Strom oder gänzlich durch die Heizpatrone kompensiert werden.

\begin{center}
\begin{minipage}{\linewidth}
\centering
\includegraphics[width=1\textwidth]{../08A.pdf}%
\captionof{figure}{Der Graph zeigt den Temperaturverlauf an Spitze und Reservoir der Jetdüse zu höheren Heizströmen. Im ersten Abschnitt des Graphs wurde der Temperaturanstieg untersucht. Die Höchsttemperatur beträgt bei dieser Konfiguration 164 °C an der Spitze und 170 °C am Reservoir. Der zweite Abschnitt löst die Abkühlung der Düse auf.}
 \label{fig:Tempverlauf8A}
\end{minipage} 
\end{center} 

%Positionsbestimmung a lá Vakuumjetexpansion gut. seite 35