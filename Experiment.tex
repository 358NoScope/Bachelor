%%%%%%%%%%%%%%%%%%%%%%%%%%%%%%%%%%%%%%%%%%%%%%%%%%%%%%%%%%%%%%%%%%%%%%%%%
%%%%%%%%%%%%%%%%%%%%%%%%%%%%%%% Experiment  %%%%%%%%%%%%%%%%%%%%%%%%%%%%%
%%%%%%%%%%%%%%%%%%%%%%%%%%%%%%%%%%%%%%%%%%%%%%%%%%%%%%%%%%%%%%%%%%%%%%%%%
\chapter{Experiment} 

Inbetriebnahme/Charakterisierung der Quelle\\

\textbf{Aufbau kann ich schonmal erklären, Quadrupol auch?}

Positionsbestimmung a lá Vakuumjetexpansion gut. seite 35

Mal schaun was ich bekomme. Zuerst hab ich nur die Drücke der Jetstages zu Verfügung. Wenn ich irgendwie sicher sein kann, dass das ganze sauber funktioniert vielleicht mit REMI ausprobieren um zu sehen welche Cluster man denn bekommt?

Sicher sein heißt, dass die Drücke alle im \enquote{richtigen} Bereich bleiben und stabil sind?

\section{Weiß noch nicht was da so bei rumkommt, aber denke dass man da erste Drücke auslesen kann. Evtl steht ein Quadrupol-Spektrometer zur verfügung mit dem man das was rasukommt analysieren kann}