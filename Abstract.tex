\begin{center}\large\sffamily{\textbf{Abstract}}\end{center}
%
%The creation of a supersonic gas jet occurs at the adiabatic expansion of a gas from a region with high pressure ($p_R \approx 1 $ bar) into a region with low pressure ($p_B \approx 10^{-6}$ bar). In this process almost all the undirected thermal velocity of the gas particles is translated into directed velocity through \textbf{impacts} amongst the gas particles. This leads to the production of a cold gas beam which can serve as target for impact spectroscopy in reaction microscopes.
%In the course of this bachelor thesis a new cluster source, which is able to produce water dimers, was developed with regard to future researches of proton transfer dynamics at FLASH in Hamburg. Aside from the comissioning of the new cluster source, results of past experiments with water clusters will be presented.
This thesis deals with the design and the assembly of a new cluster source which is intended to produce water clusters that can be used as a target in a reaction microscope. The production of these clusters is based on thermodynamical supersonic expansion. Therefore, gas is expanded adiabatically from a region of high pressure ($p_R \approx 1 $ bar) into a region of low pressure ($p_B \approx 10^{-6}$ bar). Besides the production of clusters, one gets an intrinsically cold gas target that is suited to do momentum spectroscopy with a reaction microscope. The new cluster source will allow to produce water dimers which are needed for experiments on the dynamics of proton transfer processes performed with the reaction microscope at the free-electron laser in Hamburg (FLASH).





\begin{center}\large\sffamily{\textbf{Zusammenfassung}}\end{center}
%
%Zur Erzeugung eines Überschallgasjets wird Gas aus einem Reservoir mit hohem Druck ($p_R \approx 1 $ bar) durch eine Düse in einen Bereich mit geringem Druck ($p_B \approx 10^{-6}$ bar) adiabatisch expandiert. Dabei wird die nahezu die Gesamte ungerichtete thermische Geschwindigkeit der Gasteilchen im Bereich der Düse durch Stöße in eine gerichtet Geschwindigkeit übersetzt. Der so erzeugte kalte Gasstrahl kann als Target für Impulsspektroskopie in Reaktionsmikoskopen verwendet werden. Im Rahmen dieser Bachelorarbeit wurde eine neue Cluster-Quelle entwickelt, die in Hinsicht auf zukünftige Untersuchungen von Protonentransferprozessen an FLASH in Hamburg, geeignet ist Wasserdimere zu produzieren. Neben der Inbetriebnahme dieser neuen Cluster-Quelle werden auch Ergebnisse von vergangenen Experimenten mit Wasserclustern vorgestellt.
Im Rahmen dieser Bachelorarbeit wurde eine neue Cluster-Quelle entwickelt und gebaut, die es ermöglicht Wassercluster zur Verwendung in einem Reaktionsmikroskop herzustellen. Die Erzeugung dieser Cluster beruht auf dem thermodynamischen Effekt der Überschallexpansion, bei dem Gas aus einem Reservoir hohen Drucks ($p_R \approx 1 $ bar) durch eine kleine Öffnung in einen Bereich geringen Drucks \linebreak ($p_B \approx 10^{-6}$ bar) adiabatisch expandiert wird. Desweiteren erhält man hierdurch ein kaltes Gastarget, welches für Impulsspektroskopie mit einem Reaktionsmikroskop geeignet ist. Mit der neuen Cluster-Quelle wird es möglich sein Wasserdimere herzustellen, die für Untersuchungen der Dynamik von Protonentransferprozessen mit dem Reaktionsmikroskop am Freie-Elektronen-Laser in Hamburg (FLASH) benötigt werden. 

\clearpage
\thispagestyle{empty}