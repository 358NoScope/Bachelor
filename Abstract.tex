\begin{center}\large\sffamily{\textbf{Abstract}}\end{center}

The creation of a supersonic gas jet happens at the adiabatic expansion of a gas from a region with high pressure ($p_R \approx 1 $ bar) into a region with low pressure ($p_B \approx 10^{-6}$ bar). In this process almost all the undirected thermal velocity of the gas particles is translated into directed velocity through \textbf{impacts} amongst the gas particles. This leads to the production cold gas beam which can serve as target for impact spectroscopy in reaction microscopes.
In the course of this bachelor thesis a new cluster source which is able to produce water dimers was developed with regard to researches of proton transfer dynamics at FLASH in Hamburg. Aside from the comissioning of the new cluster source, results of past experiments with water clusters will be presented.

\begin{center}\large\sffamily{\textbf{Zusammenfassung}}\end{center}

Zur Erzeugung eines Überschallgasjets wird Gas aus einem Reservoir mit hohem Druck ($p_R \approx 1 $ bar) durch eine Düse in einen Bereich mit geringem Druck ($p_B \approx 10^{-6}$ bar) adiabatisch expandiert. Dabei wird die nahezu die Gesamte ungerichtete thermische Geschwindigkeit der Gasteilchen im Bereich der Düse durch Stöße in eine gerichtet Geschwindigkeit übersetzt. Der so erzeugte kalte Gasstrahl kann als Target für Impulsspektroskopie in Reaktionsmikoskopen verwendet werden. Im Rahmen dieser Bachelorarbeit wurde eine neue Cluster-Quelle entwickelt, die in Hinsicht auf Untersuchungen von Protonentransferprozessen an FLASH in Hamburg, geeignet ist Wasserdimere zu produzieren. Neben der Inbetriebnahme dieser neuen Cluster-Quelle werden auch Ergebnisse von vergangenen Experimenten mit Wasserclustern vorgestellt.

\newpage
